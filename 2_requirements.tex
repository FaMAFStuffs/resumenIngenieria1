\section{Requerimientos de Software}
  
  La \textit{SRS} \textbf{especifica} lo que el sistema propuesto debe hacer.
  \PN Establece bases de un acuerdo entre el \textit{cliente / usuario} y el desarrollador.

  % wrapfigure is for wraping the text and get a good format. Can be omited
  \begin{wrapfigure}[5]{r}{0pt}
    \includegraphics[scale=0.3]{graphics/figure_2.png}
  \end{wrapfigure}

  \PN\DEF{Pasos para crear una \textit{SRS}}
  \begin{enumerate}
    \item \textbf{Análisis del problema o requerimientos.}
    \item \textbf{Especificación de los requerimientos.}
    \item \textbf{Validación.}
  \end{enumerate}

  \subsection{Análisis del problema o requerimientos.}

    \DEF{Objetivo} Lograr una buena comprensión de las necesidades, requerimientos y restricciones del software. Trata
    con el dominio del problema. Se utilizan técnicas de diagramas de flujo de datos (\textit{DFD}), diagramas de
    objetos, etc.
    \PN \textbf{Principio básico:} \textit{Divide y conquistarás} para comprender cada subproblema y la
    relación entre ellos con respecto a:
    \begin{enumerate}[-]
      \item \textbf{Funciones} (Análisis estructural)
      \item \textbf{Objetos} (Análisis objetos)
      \item \textbf{Eventos del sistema} (particionado de eventos)
    \end{enumerate}

    \subsubsection{Modelado de flujo de datos}
      Un \textit{DFD} es una representación del flujo de datos a través del sistema.
      \begin{itemize}
        \item Ve al sistema como una trandormación de \textit{I / O}
        \item La transformación se realiza a través de ``\textit{Transformadores}"
        \item Captura la manera en que ocurre la transformación de la entrada en la salida a medida que los datos se
              mueven a través de los transformadores.
        \item No se limita al software.
      \end{itemize}

    \subsubsection{Modelado orientado a objetos}
      Un sistema es visto como un conjunto de objetos interactuando entre
      sí (o con el usuario) a través de servicios que cada uno provee.
      \PN\DEF{Objetivo}
      \begin{itemize}
        \item Identificar los objetos en el dominio del problema
        \item Definir las clases identificando cual es la información del estado que esta encapsula
        \item Identificar las relaciones entre los objetos de las distintas clases, ya sea en la jerarquía o a través de
              llamadas a métodos.
      \end{itemize}

    \PN\subsubsection{Prototipado}
      Clientes, usuarios y desarrolladores lo utilizan para comprender mejor el problema y las necesidades.
      Dos enfoques:
      \begin{enumerate}
        \item \textbf{Descartable:} (\textit{más adecuado}) El prototipo se construye con la idea de desecharlo luego de
              culminada la fase de requerimientos.
        \item \textbf{Evolucionario:} Se contruye con la idea de que evolucionará al sistema final.
      \end{enumerate}

  \subsection{Especificación de requerimientos:}
    \subsubsection{Características de una SRS}
    \begin{itemize}
      \item \textbf{Correcta:}  Cada requerimiento representa precisamente alguna característica deseada en
                                el sistema final.
      \item \textbf{Completa:} Todas las características deseadas están descritas.
      \item \textbf{No ambigüa} Cada requerimiento tiene exactamente un significado.
      \item \textbf{Consistente} Ningún requerimiento contradice a otro.
      \item \textbf{Verificable} Cada requerimiento se debe poder verificar.
      \item \textbf{Rastreable} Se debe poder determinar el origen de cada requerimiento y cómo éste se relaciona a los
                                elementos del software. \\
                                Requerimiento $\Rightarrow$ parte de código, 
                                Parte de código $\Rightarrow$ Requerimiento.
      \item \textbf{Modificable} Incorporar cambios fácilmente preservando completitud y consistencia. No redundancia.
      \item \textbf{Ordenada en aspectos de importancia y estabilidad} Orden de prioridades para reducir riesgos debido
                                                                       a cambios de requerimientos.
    \end{itemize}

    \subsubsection{Una SRS debe especificar requerimientos sobre}
    \begin{enumerate}
      \item \textbf{Funcionalidad:} Especifica toda la funcionalidad que el sistema debe proveer, las salidas que debe
        producir para cada entrada y las relaciones entre ellas, todas las operaciones que el sistema debe realizar. Las
        entradas válidas y el comportamiento del sistema para entradas inválidas, errores u otras situaciones anormales.
      \item \textbf{Requerimientos de desempeño:} Todas las restricciones en el desempeño del sistema de software.
            \begin{enumerate}[-]
              \item \textbf{Requerimientos dinámicos:} Especifican restricciones sobre la ejecución.
              \item \textbf{Requerimientos estáticos:} No imponen restricción en la ejecución.
            \end{enumerate}
            Todos los requisitos se especifican en términos medibles, verificables.
      \item \textbf{Restricciones de diseño:} Existen factores en el entorno del cliente que pueden restringir las
        elecciones de diseño. Ej: Ajustarse a estándares.
      \item \textbf{Requerimientos de interfaces externas:} Todas las interacciones del software con gente, harware y
        otros softwares deben especificarse claramente. La interfaz con el usuario debe recibir la atención adecuada.
    \end{enumerate}

    \subsubsection{Casos de uso}

    \PN\DEF{Conceptos básicos}
    \begin{enumerate}[-]
      \item \textbf{Actor:} Persona o sistema que interactua con el sistema propuesta para alcanzar un objetivo.
      \item \textbf{Actor primario:} Actor principal que inicia el caso de uso.
      \item \textbf{Escenario:} Conjunto de acciones realizadas con el fin de alcanzar un objetivo bajo determinadas
        condiciones.
      \item \textbf{Escenario exitoso principal:} Cuando todo funciona normalmente y se alcanza el objetivo.
      \item \textbf{Escenario alternativos (de extensión/ de excepción):} Cuando algo sale mal y el objetivo no puede
        ser alcanzado.
    \end{enumerate}

  \subsection{Validación}

    \subsubsection{Errores más comunes}
      \begin{itemize}
        \item Omisión
        \item Inconsistencia
        \item Hechos incorrectos
        \item Ambigüedad
      \end{itemize}
      La \textit{SRS} la revisan un grupo de personas, conformado por: Autor, cliente, representantes de usuarios y de 
      desarrolladores.

  \subsection{Métricas}
    Para poder estimar costos y tiempos y planear el proyecto se necesita ``medir" el esfuerzo que demandará.

    \subsubsection{Punto función}
      Es una métrica como las \textit{LOC}, se determina solo cno la \textit{SRS}, define el tamaño en términos de
      funcionalidad

      \DEF{Tipos de Funciones}
      \begin{itemize}
        \item Entradas externas
        \item Salidas externas
        \item Archivos de interfaz externa
        \item Archivos lógicos internos
        \item Transacciones externas
      \end{itemize}

      \DEF{Punto función no ajustado (UFP)} \quad $\sum\limits_{i=1}^{5}\sum\limits_{j=1}^{3}W_{ij}C_{ij}$

      \DEF{Características}
        \par 1. Comunicación de datos
        \par 2. Procesamiento distribuido
        \par 3. Objetivos de desempeño
        \par 4. Carga en la configuración de operación
        \par 5. Tasa de transacción
        \par 6. Ingreso de datos online
        \par 7. Eficiencia del usuario final
        \par 8. Actualización online
        \par 9. Complejidad del procesamiento lógico
        \par 10. Reusabilidad
        \par 11. Facilidad para la instalación
        \par 12. Facilidad para la operación
        \par 13. Múltiples sitios
        \par 14. Intención de facilitar cambios

      \DEF{Factor de ajuste de complejidad (CAF)} \quad $0.65 + 0.01 \sum\limits_{i=1}^{14} Pi$

      \DEF{Puntos función} CAF * UFP

    \subsubsection{Métrica de calidad}
      \begin{enumerate}[-]
        \item Directa: Evalúan la calidad del documento estimando el valor de los atributos de calidad de la SRS.
        \item Indirecta: Evalúan la efectividad de las métricas del control de calidad usadas en el proceso en la fase
              de requerimientos.
      \end{enumerate}