\section{Arquitectura del software}
  \PN\DEF{Definición} Es la estructura del sistema que comprende los elementos del software, las propiedades
  externamente visibles de tales elementos y la relación entre ellos.

  \begin{itemize}
    \item Diseño de mas alto nivel.
    \item Elecciones de tecnologias, productos a usar, servidores.
    \item Empezar a evaluar confiabilidad y desempeño.
    \item Divide al sistema en partes lógicas independientes.
  \end{itemize}

  \subsection{El rol de la arquitectura}

    \subsubsection{Comprensión y comunicación}
      \begin{enumerate}[-]
        \item Al mostrar la estructura de alto nivel del sistema, la descripción arquitectónica facilita la
              comunicación.
        \item Define un marco de comprensión común entre los distintos interesados
              (usuarios, cliente, arquitecto, diseñador, etc.).
      \end{enumerate}

    \subsubsection{Reuso}
      \begin{enumerate}[-]
        \item Principal técnica para incrementar la productividad.
        \item Una forma de reuso es componer el sistema con partes existenes, reusadas, otras nuevas.
        \item La arquitectura es muy importante: se elige una arquitectura tal que las componentes existentes encajen 
              adecuadamente con otras componentes a desarrollar.
        \item Las decisiones sobre el uso de componentes existentes se toman en el momento de diseñar la arquitectura.
      \end{enumerate}

    \subsubsection{Construcción y evolución}
      \begin{enumerate}[-]
        \item La división provista por la arquitectura servirá para guiar el desarrollo del sistema.
        \item Decide las partes a cambiar para incorporarse nuevas características en la evolución del software.
      \end{enumerate}

    \subsubsection{Análisis}
      \begin{enumerate}[-]
        \item Permite considerar distintas alternativas de diseño hasta encontrar los niveles de satisfacción deseados.
      \end{enumerate}

  \pagebreak

  \subsection{Vistas de la arquitectura}

    \subsubsection{Módulos}
      \begin{enumerate}[-]
        \item Un sistema es una colección de unidades de código
        \item La relación entre ellos esta basada en el código.
      \end{enumerate}

    \subsubsection{Componentes y conectores}
      \begin{enumerate}[-]
        \item Los elementos son entidades de ejecucción  (componentes).
        \item Los conectores proveen el medio de interacción entre los componentes.
      \end{enumerate}

    \subsubsection{Asignación de recursos}
      \begin{enumerate}[-]
        \item Como las unidades de \textit{SW} se asignan a \textit{SW}.
        \item Exponen propiedades estructurales. Ej: que archivo reside donde.
      \end{enumerate}

  \subsection{La vista componentes y conectores}
    
    \PN\DEF{Componentes} Elementos computacionales o de almacenamiento de datos.

    \PN\DEF{Conectores} Mecanismos de interacción entre las componentes.

    \PN La vista de C\&C describe una estructura en ejecucción del sistema: que componentes existen y como interactúan
    entre ellos en tiempo de ejecucción.

  \subsection{Estilos arquitectónicos para la vista C\&C}

    \begin{itemize}
      \setlength\itemsep{0mm}
      \item Un estilo arquitectónico define una gamilia de arquitecturas que satisface las restricciones de ese estilo.
      \item Los estilos proveen ideas para crear arquitecturas de sistemas.
      \item Distintos estilos pueden conbinarse para definir una nueva arquitectura.
    \end{itemize}

  \subsection{Tubos y filtros}

    \begin{itemize}
      \setlength\itemsep{0mm}
      \item Adecuado para sistemas que realizan transformación de datos
      \item Un solo tipo de componente: \textbf{FILTRO}
      \item Un solo tipo de conector: \textbf{TUBO}
    \end{itemize}

    \subsubsection{Filtro}
    \begin{enumerate}[-]
      \setlength\itemsep{0mm}
      \item Realiza transformaciones y le pasa los datos a otro filtro.
      \item Entidad independiente y asincrona.
      \item No necesita saber la identidad de los filtros a los que envía y recibe datos.
      \item Se encarga de hacer ``buffering".
    \end{enumerate}

    \subsubsection{Tubo}
    \begin{enumerate}[-]
      \setlength\itemsep{0mm}
      \item Canal unidireccional que transporta un flujo de datos de un filtro a otro
      \item Solo conecta dos componentes
    \end{enumerate}

    %TODO: Ejemplo

  \subsection{Datos compartidos}

    \subsubsection{Componentes}
    \begin{enumerate}[-]
      \setlength\itemsep{0mm}
      \item Repositorio de datos: Provee almacenamiento permanente confiable.
      \item Usuarios de datos: Acceden a los datos en el repositorio, realizan cálculos y ponen los resultados otra vez
      en el repositorio. La comunicación entre los usuarios de los datos sólo se hace a través del repositorio.
    \end{enumerate}

    \subsubsection{Conector}
    \begin{enumerate}[-]
      \setlength\itemsep{0mm}
      \item Lectura / escritura
    \end{enumerate}

    \subsubsection{Variantes}
    \begin{enumerate}[-]
      \item Estilo pizarra: Cuando se agregan/modifican datos en el repositorio se informa a todos los usuarios.
      \item Estilo repositorio: El repositorio es pasivo.
    \end{enumerate}

  \subsection{Estilo cliente-servidor}

    \subsubsection{Componentes}
    \begin{enumerate}[-]
      \item 2 tipos: Clientes y servidores
      \item Los clientes sólo se comunican con el servidor, y no con otros clientes.
      \item La comunicación es iniciada por el cliente, envia una solicitud al servidor y espera una respuesta.
      \item Comunicación usualmente sincronica.
      \item Usualmente cliente y servidor en distintas máquinas.
    \end{enumerate}

    \subsubsection{Conector}
    \begin{enumerate}[-]
      \item Solicitud / respuesta, es asimétrico
    \end{enumerate}

    %TODO: Ejemplo

  \subsection{Estilo publicar-suscribir}
  
    \subsubsection{Componentes}
    \begin{enumerate}[-]
      \item Dos tipos de componentes: las que publican eventos y las que se suscriben a eventos.
            Cada vez que un evento es publicado se invoca a las componentes suscriptas a dicho evento.
    \end{enumerate}

  \subsection{Estilo peer-to-peer}

    \begin{enumerate}[-]
      \item Un único tipo de componente.
      \item Cada componente le puede pedir servicios a otro modelo de computación orientado a objetos.
    \end{enumerate}

  \subsection{Estilo de procesos que se comunican}

    \begin{enumerate}[-]
      \item Procesos que se comunican entre sí a través de pasaje de mensajes.
    \end{enumerate}

  \subsection{Relacion/diferencia entre diseño y arquitectura}

    \begin{enumerate}[-]
      \item La arquitectura es un diseño: se encuentra en el dominio de la solución y no en el dominio del problema.
      \item La arquitectura es un diseño de muy alto nivel que se enfoca en las componentes principales.
      \item Lo que usualmente llamamos diseño se enfoca en los módulos que finalmente se transformarán en el código de
            tales componentes.
    \end{enumerate}

  \subsection{Preservación de la intergridad de la arquitectura}

    \begin{enumerate}[-]
      \item La arquitectura impone restricciones que deben preservarse en la implementación, inclusive.
      \item Para que la arquitectura tenga sentido, ésta debe acompañar el diseño y el desarrollo del sistema.
    \end{enumerate}

  \subsection{Método de análisis ATAM}
    Analiza las propiedades y las conexiones entre ellas

    \begin{enumerate}
      \item Recolectar escenarios
            \begin{itemize}
              \item Elegir los escenarios de intereces para el analisis
              \item incluir escenarios exxcepcionales solo si son importantes
            \end{itemize}
      \item Recolectar requerimientos y/o restricciones
            \begin{itemize}
              \item Definir lo que se espera del sistema en tales escenarios.
              \item Deben especificar los niveles deseados para los atributos de interés (preferiblemente
                    cuantificados).
            \end{itemize}
      \item Describir las vistas arquitectónicas
            \begin{itemize}
              \item Las vistas del sistema que serán evaluadas son recolectadas.
              \item Distintas vistas pueden ser necesarias para distintos análisis.
            \end{itemize}
      \item Análisis especificos a cada atributo
            \begin{itemize}
              \item Se analizan las vistas bajo distintos escenarios separadamente para cada atributo de interés
                    distinto.
              \item Determina los niveles que la arquitectura puede proveer en cada atributo.
              \item Se comparan con los requeridos.
              \item Esto forma la base para la elección entre una arquitectura u otra o la modificación de la
                    arquitectura propuesta.
            \end{itemize}
      \item Identificar puntos sensitivos y de compromisos
            \begin{itemize}
              \item Análisis de sensibilidad: cuál es el impacto que tiene un elemento sobre un atributo de calidad.
                    Los elementos de mayor impacto son los puntos de sensibilidad.
              \item Análisis de compromiso: Los puntos de compromiso son los elementos que son puntos de sensibilidad
                    para varios atributos. 
            \end{itemize}
    \end{enumerate}